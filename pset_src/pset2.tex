\documentclass{pset}

\name{Arjun Biddanda}
\date{7/7/2025}
\psnum{2}
\class{Statistical Genetics Reading Group}

\begin{document}
\maketitle

\section*{GWAS Power and Study Design}

\subsubsection*{Sample Size Calculations}

Suppose you are conducting a GWAS for quantitative trait $YFT$ (your favorite quantitative trait), and you want to collect $N=100,000$ individuals. For all of the calculations use the genome-wide threshold of $\alpha = 5\times 10^{-8}$ when determining power.

\begin{itemize}
	\item At a power threshold of $0.8$, what is the minimal standardized effect size ($\beta$) detectable for a fully typed variant $r^2 = 1$ at an MAF of: 
	\begin{itemize}
		\item 5\% 
		\item 1\%
		\item 0.1\%
	\end{itemize}
	\item Your colleague wants to see if a specific missense variant at 0.1\% frequency in the population (with an imputation $r^2 = 0.8$), would be detectable in your current GWAS design?. What would be the minimal effect-size detectable for this variant?  
\end{itemize}

\subsubsection*{$\bLozenge\bLozenge$ GWAS Discovery Rate} 

\begin{enumerate}
\item Fixing GWAS power at $0.8$ and assuming fully typed variants ($r^2 = 1)$, if causal variant effects are drawn from $\beta_{causal} \sim \mathcal{N}(0,\sigma^2)$, what is the \textit{expected fraction of additional discoveries} made in going from $N_1 = 10^4$ to $N_2 = 10^5$ when $\sigma^2 = 2$? (Assume causal allele frequencies are drawn from the Uniform distribution). 
	
\item If $\beta | p \sim N(0, \sigma^2)$, where $\sigma^2 = \left(p(1-p)\right)^{\alpha}$ for causal variants (e.g. the effect size is linked to allele frequency), what is the expected fraction of additional discoveries when $\alpha = -0.75$? (assume causal allele frequencies are drawn from the Uniform distribution).
\end{enumerate}

Provide either an analytical solution or a plot reflecting the numerical solution for both scenarios. 

\subsubsection*{$\bLozenge\bLozenge\bLozenge$  Effect of Tagging Variant LD}

The $r^2$ measure of LD is critical when you do not have direct access to a causal variant and only have a tagging variant. If we assume that $p_{tag} >= p_{causal}$, the expression is: 

$$r^2_{max}(tag, causal) = \frac{(1 - p_{tag})(1 - p_{causal})}{p_{tag}p_{causal}}$$

Using the expressions for GWAS power for a quantitative trait, and $r^2_{max}(p_{tag}, p_{causal})$, show that:

$$Power_{tag} \geq Power_{causal} \hspace{1em}\forall \hspace{1em} p_{tag} \geq p_{causal}$$

\subsubsection*{$\bLozenge\bLozenge$ Comparing GWAS}

Your colleague has an interesting case where $YFT$ is highly prevalent in a different population (lets say population ``B'')and wants to understand if the same variant is driving this increase in prevalence. Your collaborator has access to $N_B=50000$ samples, but doesn't have a good sense of whether this would be well-suited to address the idea.

If the effect-size of the causal variant is the same (e.g. $\beta_A = \beta_B$), what would be the difference in frequency of the causal variant required to maintain a power of 80\% if the variant in population $A$ is at 0.1\% frequency? (Assume $N_A = 100,000$)  


\section*{GWAS Practical}

For a GWAS practical, we will use publicly available summary statistics from the Pan-UKBB project. Specifically we will use data for a dataset on total bilirubin (a key liver enzyme) where GWAS was performed separately across six different ancestries in the UK Biobank. The ancestral groupings are: 

\begin{itemize}
	\setlength{\itemsep}{0pt}
	\item EUR = European ancestry($N=399286$)
	\item CSA = Central/South Asian ancestry ($N=8395$)
	\item AFR = African ancestry ($N=6176$)
	\item EAS = East Asian ancestry ($N=2549$)
	\item MID = Middle Eastern ancestry ($N=1491$)
	\item AMR = Admixed American ancestry ($N=933$)
\end{itemize}

To obtain the data for this exercise run: 

\begin{center}
\texttt{wget \url{https://pan-ukb-us-east-1.s3.amazonaws.com/sumstats_flat_files/biomarkers-30840-both_sexes-irnt.tsv.bgz}}
\end{center}

\subsection*{Genomic Inflation Factor}

While many covariates have been accounted for when evaluating genetic effects - it is possible that there are still unobserved confounders present. One way to  evaluate the extent of possible confounding has been using the genomic-inflation factor, which is defined as: 

$$\hat{\lambda}_{GC} = \text{median}(\hat{\chi}^2) / F^{-1}_{\chi^2_1}(0.5)$$  

where $F^{-1}_{\chi^2_1}(0.5) \approx 0.455$ is the expected median of the chi-squared distribution with 1 degree of freedom under the null hypothesis of no association. 

\begin{enumerate}
\item Estimate $\hat{\lambda}_{GC}$ for each individual population separately using the relationship that $\hat{\chi}^2 = \left(\frac{\hat{\beta}}{s_\beta}\right)^2$. 
\item Are the estimates of $\lambda_{GC}$ significantly $> 1$ for each population? (Hint: use bootstrapping to re-sample the effect-sizes to construct empirical 95\% confidence intervals)
\item $\bLozenge$ Re-estimate $\lambda$ using only every 10000$^{th}$ variant to avoid the effect of linkage disequilibrium and evaluate the differences relative the non LD-pruned version. Comment on how LD may violate some of the assumptions of $\lambda_{GC}$.
\end{enumerate} 

\subsection*{Power \& Shared Signals}

One of the useful things about the Pan-UKBB project is that it can illustrate how to compare the \textit{power} of GWAS across different study designs and sample-populations.

\begin{enumerate}
\item For a start, let us compare the intersection of signals found at a genome-wide threshold $\alpha = 5\times 10^{-8}$ in the EUR and CSA sample subsets? Plot the fraction of signals that overlap as a function of $\alpha \in \{10^{-4}, 10^{-20}\}$
\item Since we have the allele-frequency of the variants in each ancestry subset, for all shared marginal signals at the detection threshold $\alpha$, estimate the power to detect each of the shared signals? What is the mean detection power threshold for shared signals at different values of $\alpha$? (NOTE: assume that $\hat{\beta} \approx \beta_{causal}$ and $r^2 = 1$ for simplicity, though consider how violations of these assumptions might alter results) 
\end{enumerate}


\end{document}
