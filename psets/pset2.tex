\documentclass{pset}

\name{Arjun Biddanda}
\date{6/1/2025}
\psnum{2}
\class{Statistical Genetics Reading Group}

\begin{document}
\maketitle

\section{Polygenicity}

One core aspect of complex trait architectures is the total number of loci ($M$) involved in the etiology of a complex trait. The proportion of \textit{non-null} genetic associations can be thought of as the polygenicity of a complex trait. We will consider the following simple model for polygenicity based on a normal mixture model:  

$$Z \sim \pi_0 \mathcal{N}(0,1) + (1 - \pi_0)\mathcal{N}(0,\sigma)$$

\subsection*{Likelihood Calculation}


\subsection*{Maximum-Likelihood Estimation}

Using the summary statistics provided - calculate the maximum-likelihood estimates of $\hat{\Theta} = (\hat{\pi}_0, \hat{\sigma})$ for trait A and trait B. You may use numerical or analytical methods. 

\subsection*{Revised Model $\bLozenge\bLozenge$}

Let us consider a revised model where we directly model the estimated univariate effect-sizes and their standard errors $(\hat{\beta}, \hat{s})$:

$$f(\hat{\beta} | \beta, \hat{s}) = P(\hat{\beta} | \beta)$$



\subsection*{Estimation from Empirical Datasets}

Using summary statistics from the Pan-UKBB for a number of 




\section{Winner's Curse}

% NOTE: this is going to be based on 



\section{Heritability}




\section{Indirect Genetic Effects}




\end{document}
